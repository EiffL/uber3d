\documentclass[a4paper,11pt]{article}
\pdfoutput=1 % if your are submitting a pdflatex (i.e. if you have
             % images in pdf, png or jpg format)

\usepackage{jcappub} % for details on the use of the package, please
                     % see the JCAP-author-manual

\usepackage[T1]{fontenc} % if needed

\usepackage[]{graphicx}
\usepackage{url}
\usepackage{hyperref}
\hypersetup{breaklinks,colorlinks,citecolor=blue}
\usepackage{setspace}
\usepackage{amssymb}
\usepackage{amsfonts}
\usepackage{amsmath}
\usepackage{mathtools}
\usepackage{color}


\title{\boldmath Uber3D: spherical 3D transforms}

%% %simple case: 2 authors, same institution

\author{Franz~Elsner,}
\author{Francois~Lanusse,}
\author{Boris~Leistedt,}
\author{Jason~D.~McEwen,}
\author{Hiranya~V.~Peiris,}
\author{Layne~Price,}

\author[1]{and\ldots\ \textcolor{red}{add your name here}\note{Please add names in alphabetical order.  If it becomes appropriate to turn this document into a paper then we will re-address author ordering then.}}
\affiliation{\today}

% % more complex case: 4 authors, 3 institutions, 2 footnotes
% \author[a,b,1]{F. Irst,\note{Corresponding author.}}
% \author[c]{S. Econd,}
% \author[a,2]{T. Hird\note{Also at Some University.}}
% \author[a,2]{and Fourth}
% % The "\note" macro will give a warning: "Ignoring empty anchor..."
% % you can safely ignore it.
% \affiliation[a]{One University,\\some-street, Country}
% \affiliation[b]{Another University,\\different-address, Country}
% \affiliation[c]{A School for Advanced Studies,\\some-location, Country}

% e-mail addresses: one for each author, in the same order as the authors
\emailAdd{francois.lanusse@cea.fr}
\emailAdd{boris.leistedt@gmail.com}
\emailAdd{jason.mcewen@ucl.ac.uk}
\emailAdd{h.peiris@ucl.ac.uk}

% Macros for equations
\newcommand{\naturals}{\ensuremath{{\mathbb{N}}}}
\newcommand{\integers}{\ensuremath{{\mathbb{Z}}}}
\newcommand{\reals}{\ensuremath{{\mathbb{R}}}}
\newcommand{\realsnn}{\ensuremath{{\mathbb{R}^{+}}}}
\newcommand{\ball}{\ensuremath{{\mathbb{B}^3}}}
\newcommand{\sphere}{\ensuremath{{\mathbb{S}^2}}}
\newcommand{\dx}{\ensuremath{\mathrm{\,d}}}
\newcommand{\im}{{\rm i}}
\newcommand{\rvec}{{\boldsymbol{r}}}
\newcommand{\rang}{{\boldsymbol{\hat{r}}}}
\newcommand{\rlen}{{r}}
\newcommand{\spinup}{\ensuremath{\eth}}
\newcommand{\spindown}{\ensuremath{\bar{\eth}}}


% Commands to highlight comments
\newcommand{\todo}[1]{\textcolor{red}{[TO DO: #1]}}
\newcommand{\jdm}[1]{\textcolor{red}{[JDM: #1]}}




\abstract{Abstract...}



\begin{document}
\maketitle
\flushbottom


\section{Introduction}


\newpage
\section{Basis functions}

We define the spherical basis functions and the conventions adopted explicitly in this section.  


\subsection{Domains}

We consider the space formed by augmenting the sphere with the real half-line to form the ball $\ball = \realsnn \times \sphere$, where the real half-line is denoted $\realsnn = [0, \infty)$ and the sphere $\sphere$.  Throughout we consider spherical coordinates with $\rvec = \rlen \rang = \rvec(\rlen,\theta,\phi) \in \ball$, where $\rlen \in \realsnn$ and $\rang=(\theta,\phi) \in \sphere$, with colatitude $\theta\in[0,\pi]$ and longitude $\phi \in [0, 2\pi)$. The standard invariant measure on $\ball$ reads $\dx^3 \rvec = \rlen^2 \dx \rlen \dx \Omega(\rang) =  \rlen^2 \sin\theta \dx \rlen \dx \theta \dx \phi$, where $\dx \Omega(\rang)=\sin\theta \dx \theta \dx \phi$ is the invariant measure on the sphere.


\subsection{Spherical Bessel functions}



\subsubsection{Definition}

The spherical Bessel functions are defined by
\begin{equation}
  j_\ell(x) = \sqrt{\frac{\pi}{2x}} \: J_{\ell+1/2}(x)
  ,
\end{equation}
for $x\in \reals$, natural $\ell \in \naturals$, with $\reals=(-\infty,\infty)$, where $J_\ell(\cdot)$ are the Bessel functions of the first kind.


\subsubsection{Closure}

The spherical Bessel functions satisfy the following closure relation,
\begin{equation}
  \int_{\realsnn} \dx x x^2 \: j_\ell(ux) \: j_\ell(vx) 
  =
  \int_0^\infty \dx x x^2 \: j_\ell(ux) \: j_\ell(vx) 
  =
  \frac{\pi}{2u^2} \: 
  \delta(u-v)
  ,
\end{equation}
for $u,v \in \realsnn$, which plays the role of both standard orthogonality and completeness relations.


\subsection{Spherical Laguerre functions}

\todo{To add\ldots}

\subsection{Spherical harmonic functions}


\subsubsection{Definition}

The scalar spherical harmonic functions are defined by
\begin{equation}
  Y_{\ell m}(\theta, \phi) = 
  \sqrt{\frac{4\pi}{2\ell+1} \frac{(\ell-m)!}{(\ell+m)!}} \:
  P_\ell^m(\cos\theta) \:
  \exp{(\im m \phi)}
  ,
\end{equation}
for natural $\ell \in \naturals$, integer $m \in \integers$, $\vert m \vert \leq \ell$, where $P_\ell^m(\cdot)$ are the associated Legendre functions.

Spin raising and lowering operators, $\spinup$ and $\spindown$ respectively, increment and decrement the spin order of a spin $s \in \integers$ function by unity and are defined by
\begin{equation}  
  \spinup \equiv
  -\sin^s \phi 
  \biggl ( 
  \frac{\partial}{\partial \phi} 
  + \frac{\im}{\sin\phi} \frac{\partial}{\partial \theta}
  \biggr)
  \sin^{-s}\phi
\end{equation}
and
\begin{equation}  
  \spindown \equiv 
  -\sin^{-s} \phi 
  \biggl ( 
  \frac{\partial}{\partial \phi} 
  - \frac{\im}{\sin\phi} \frac{\partial}{\partial \theta}
  \biggr)
  \sin^{s}\phi
  ,
\end{equation}
respectively.  
%
The spin-$s$ spherical harmonics can thus be expressed in terms of
the scalar (spin-zero) harmonics through the spin raising and lowering
operators by 
\begin{equation}    
  {}_s Y_{\ell m}(\theta,\phi)
  =
  \biggl[ \frac{(\ell-s)!}{(\ell+s)!} \biggr]^{1/2} 
  \spinup^s
  Y_{\ell m}(\theta,\phi)
  ,
\end{equation}
for $0 \leq s \leq \ell$,
and by 
\begin{equation}    
  {}_s Y_{\ell m}(\theta,\phi)
  =
  (-1)^s
  \biggl[ \frac{(\ell+s)!}{(\ell-s)!} \biggr]^{1/2} 
  \spindown^{-s}
  Y_{\ell m}(\theta,\phi)  
  ,
\end{equation}
for $-\ell \leq s \leq 0$.



\subsubsection{Orthogonality}

The spin spherical harmonics satisfy the orthogonality relation:
\begin{equation}
  \int_\sphere \dx \Omega(\rang) \:
  {}_s Y_{\ell m}(\rang) \:
  {}_s Y_{\ell^\prime m^\prime}^\ast(\rang) 
  = 
  \int_0^{2\pi} \dx \phi \:
  \int_0^{\pi} \dx \theta \sin\theta \:
  {}_s Y_{\ell m}(\theta, \phi) \:
  {}_s Y_{\ell^\prime m^\prime}^\ast(\theta, \phi) 
  = 
  \delta_{\ell \ell^\prime} \:
  \delta_{m m^\prime}
  .
\end{equation}

\subsubsection{Completeness}

The spin spherical harmonics satisfy the completeness relation:
\begin{equation}
  \sum_{\ell=0}^\infty \sum_{m=-\ell}^{\ell}
  {}_s Y_{\ell m}(\rang) \:
  {}_s Y_{\ell m}^\ast(\rang^\prime) 
  = 
  \delta(\rang-\rang^\prime)
  = 
  \frac{1}{\sin\theta} \:
  \delta(\theta-\theta^\prime) \:
  \delta(\phi-\phi^\prime)
\end{equation}


\subsection{Fourier-Bessel basis functions}

\subsubsection{Definition}

Spin Fourier-Bessel basis functions are formed by the produce of the spherical Bessel functions and the spin spherical harmonics:
\begin{equation}
  {}_s Z_{\ell m}(k, \rvec) 
  = \sqrt{\frac{2}{\pi}}\:
  j_\ell(k \rlen) \: {}_s Y_{\ell m}(\rang) 
  ,
\end{equation}
where $k \in \realsnn$.

\subsubsection{Orthogonality}

The (spin) Fourier-Bessel functions satisfy the following orthogonality relation:
\begin{equation}
  \int_\ball \dx^3 \rvec \:
  {}_s Z_{\ell m}(k, \rvec) \:
  {}_s Z_{\ell^\prime m^\prime}^\ast(k^\prime, \rvec)   
   = 
  \frac{1}{k^2} \:
  \delta(k-k^\prime) \:
  \delta_{\ell \ell^\prime} \:
  \delta_{m m^\prime}   
\end{equation}


\subsubsection{Completeness}

The (spin) Fourier-Bessel functions satisfy the following completeness relation:
\begin{align}  
  \sum_{\ell=0}^\infty \sum_{m=-\ell}^{\ell}
  \int_{\realsnn} \dx k k^2 \:
  & {}_s Z_{\ell m}(k, \rvec) \:
  {}_s Z_{\ell m}^\ast(k, \rvec^\prime) \\ 
  & =   
  \frac{2}{\pi} \sum_{\ell=0}^\infty \sum_{m=-\ell}^{\ell} \:
  {}_s Y_{\ell m}(\rang) \:
  {}_s Y_{\ell m}^\ast(\rang^\prime) 
  \int_{\realsnn} \dx k k^2 \: j_\ell(k \rlen) \: j_\ell(k \rlen^\prime) \\
  & = 
  \frac{\pi}{2\rlen^2} \:
  \delta(\rang-\rang^\prime) \:
  \delta(\rlen-\rlen^\prime)\\
  & = 
  \frac{\pi}{2\rlen^2\sin\theta} \:
  \delta(\theta-\theta^\prime) \:
  \delta(\phi-\phi^\prime)
\end{align}


\subsection{Fourier-Laguerre basis functions}

\todo{To add\ldots}


\section{Fourier-Bessel transforms}



\subsection{Forward transform}

The forward Fourier-Bessel transform reads:

\begin{align}
  {}_s f_{\ell m}(k) 
  & = 
  \sqrt{\frac{2}{\pi}} 
  \int_{\ball} \dx ^3 \rvec \:
  f(\rvec) \:
  j_\ell(k\rlen) \:
  {}_s Y_{\ell m}^\ast(\rang) \\
  & =  
  \int_0^{\pi}\dx \theta 
  \int_0^{2\pi} \dx \phi
  \int_0^\infty \dx \rlen 
  \rlen^2 \:
  f(\rvec) \:
  j_\ell(k\rlen) \:
  {}_s Y_{\ell m}^\ast(\rang) 
\end{align}

\subsection{Inverse transform}

The inverse Fourier-Bessel transform reads:
\begin{align}
  f(\rvec) 
  & =  
  \sqrt{\frac{2}{\pi}} 
  \sum_{\ell m}
  \int_{\realsnn} \dx k k^2 \:
  {}_s f_{\ell m}(k) \:
  j_\ell(k\rlen) \:
  {}_s Y_{\ell m}(\rang) \\
  & =
  \sqrt{\frac{2}{\pi}} 
  \sum_{\ell m}
  \int_0^\infty \dx k k^2 \:
  {}_s f_{\ell m}(k) \:
  j_\ell(k\rlen) \:
  {}_s Y_{\ell m}(\rang) 
\end{align}

\subsection{Quadrature}

\todo{To add\ldots}

\section{Fourier-Laguerre transform}







% \appendix
% \section{Some appendix}

% \acknowledgments
% This is the most common positions for acknowledgments. A macro is
% available to maintain the same layout and spelling of the heading.

\end{document}
