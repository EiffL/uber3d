\documentclass[a4paper,11pt]{article}
\pdfoutput=1 % if your are submitting a pdflatex (i.e. if you have
             % images in pdf, png or jpg format)

\usepackage{jcappub} % for details on the use of the package, please
                     % see the JCAP-author-manual

\usepackage[T1]{fontenc} % if needed

\usepackage[]{graphicx}
\usepackage{url}
\usepackage{hyperref}
\hypersetup{breaklinks,colorlinks,citecolor=blue}
\usepackage{setspace}
\usepackage{amssymb}
\usepackage{amsfonts}
\usepackage{amsmath}
\usepackage{mathtools}
\usepackage{color}


\title{\boldmath Uber3D: spherical 3D transforms}

%% %simple case: 2 authors, same institution

\author{Franz~Elsner,}
\author{Francois~Lanusse,}
\author{Boris~Leistedt,}
\author{Jason~D.~McEwen,}
\author{Hiranya~V.~Peiris,}
\author{Layne~Price,}
\author{Jean-Eric~Campagne,}

\author[1]{and\ldots\ \textcolor{red}{add your name here}\note{Please add names in alphabetical order.  If it becomes appropriate to turn this document into a paper then we will re-address author ordering then.}}
\affiliation{\today}

% % more complex case: 4 authors, 3 institutions, 2 footnotes
% \author[a,b,1]{F. Irst,\note{Corresponding author.}}
% \author[c]{S. Econd,}
% \author[a,2]{T. Hird\note{Also at Some University.}}
% \author[a,2]{and Fourth}
% % The "\note" macro will give a warning: "Ignoring empty anchor..."
% % you can safely ignore it.
% \affiliation[a]{One University,\\some-street, Country}
% \affiliation[b]{Another University,\\different-address, Country}
% \affiliation[c]{A School for Advanced Studies,\\some-location, Country}

% e-mail addresses: one for each author, in the same order as the authors
\emailAdd{francois.lanusse@cea.fr}
\emailAdd{boris.leistedt@gmail.com}
\emailAdd{jason.mcewen@ucl.ac.uk}
\emailAdd{h.peiris@ucl.ac.uk}
\emailAdd{campagne@lal.in2p3.fr}

% Macros for equations
\newcommand{\naturals}{\ensuremath{{\mathbb{N}}}}
\newcommand{\integers}{\ensuremath{{\mathbb{Z}}}}
\newcommand{\reals}{\ensuremath{{\mathbb{R}}}}
\newcommand{\realsnn}{\ensuremath{{\mathbb{R}^{+}}}}
\newcommand{\ball}{\ensuremath{{\mathbb{B}^3}}}
\newcommand{\sphere}{\ensuremath{{\mathbb{S}^2}}}
\newcommand{\dx}{\ensuremath{\mathrm{\,d}}}
\newcommand{\im}{{\rm i}}
\newcommand{\rvec}{{\boldsymbol{r}}}
\newcommand{\rang}{{\boldsymbol{\hat{r}}}}
\newcommand{\rlen}{{r}}
\newcommand{\spinup}{\ensuremath{\eth}}
\newcommand{\spindown}{\ensuremath{\bar{\eth}}}


% Commands to highlight comments
\newcommand{\todo}[1]{\textcolor{red}{[TO DO: #1]}}
\newcommand{\jdm}[1]{\textcolor{red}{[JDM: #1]}}




\abstract{Abstract...}



\begin{document}
\maketitle
\flushbottom


\section{Introduction}


\newpage
\section{Basis functions}

We define the spherical basis functions and the conventions adopted explicitly in this section.  


\subsection{Domains}

We consider the space formed by augmenting the sphere with the real half-line to form the ball $\ball = \realsnn \times \sphere$, where the real half-line is denoted $\realsnn = [0, \infty)$ and the sphere $\sphere$.  Throughout we consider spherical coordinates with $\rvec = \rlen \rang = \rvec(\rlen,\theta,\phi) \in \ball$, where $\rlen \in \realsnn$ and $\rang=(\theta,\phi) \in \sphere$, with colatitude $\theta\in[0,\pi]$ and longitude $\phi \in [0, 2\pi)$. The standard invariant measure on $\ball$ reads $\dx^3 \rvec = \rlen^2 \dx \rlen \dx \Omega(\rang) =  \rlen^2 \sin\theta \dx \rlen \dx \theta \dx \phi$, where $\dx \Omega(\rang)=\sin\theta \dx \theta \dx \phi$ is the invariant measure on the sphere.


\subsection{Spherical Bessel functions}



\subsubsection{Definition}

The spherical Bessel functions are defined by
\begin{equation}
  j_\ell(x) = \sqrt{\frac{\pi}{2x}} \: J_{\ell+1/2}(x)
  ,
\end{equation}
for $x\in \reals$, natural $\ell \in \naturals$, with $\reals=(-\infty,\infty)$, where $J_\ell(\cdot)$ are the Bessel functions of the first kind.


\subsubsection{Closure}

The spherical Bessel functions satisfy the following closure relation,
\begin{equation}
  \int_{\realsnn} \dx x x^2 \: j_\ell(ux) \: j_\ell(vx) 
  =
  \int_0^\infty \dx x x^2 \: j_\ell(ux) \: j_\ell(vx) 
  =
  \frac{\pi}{2u^2} \: 
  \delta(u-v)
  ,
\end{equation}
for $u,v \in \realsnn$, which plays the role of both standard orthogonality and completeness relations.


\subsection{Spherical Laguerre functions}
%JEC 20 July 16
The spherical Laguerre function, or hereafter ${\cal K}$-function, is defined as
\begin{align}
{\cal K}^\alpha_n(r, \tau) &\equiv \tau^{-3/2} \sqrt{\frac{n!}{\Gamma(n+\alpha)}} e^{-r/2\tau} \left( \frac{r}{\tau}\right)^{\alpha/2 -1} L_n^{(\alpha)}(r/\tau)= \tau^{-3/2}{\cal K}^\alpha_n(r/\tau,1) \quad \quad\quad  (n\in \mathbb{N}, \quad \alpha \in \mathbb{R})
\label{eq-sphLag-def}
\end{align}
 with $L_n^{(\alpha)}$ the modified Laguerre polynomial 
 \begin{equation}
L^{(\alpha)}_n(x)  = \sum_{j=0}^n \left( \begin{array}{c} n + \alpha \\ n - j \end{array}\right) \frac{(-x)^j}{j!}
\label{eq-lagpoly-def}
\end{equation}
The $\tau  \in \realsnn$ parameter in Eq.~\ref{eq-lagpoly-def} is used to rescaled the radial coordinate sampling as described in Sec.~\ref{sec-Lag-quadrature}.
\subsubsection{Orthogonality}
Thanks to the orthogonality of the modified Laguerre polynomials w.r.t the weighting function $x^\alpha e^{-x}$, then the  ${\cal K}$-functions are orthogonal w.r.t the $r^2$ measure according to 
\begin{equation}
\int_\realsnn {\cal K}^\alpha_n(r, \tau) {\cal K}^\alpha_m(r, \tau) r^2 \dx r = \delta_{nm}
\label{eq-sphLag-orto}
\end{equation}
\subsubsection{Recurrence formula}
The modified Laguerre polynomials satisfy a series of recurrence relations among them one is particularly useful, namely
\begin{equation}
L^{(\alpha)}_n(x) = \frac{1}{n}\left( (2n + \alpha -1 -x)L^{(\alpha)}_{n-1}(x) - (n +  \alpha -1)L^{(\alpha)}_{n-2}(x) \right)
\label{eq-laguerre-recu}
\end{equation}
with the starting seeds $L^{(\alpha)}_0(x)= 1$ and $L^{(\alpha)}_1(x)=(\alpha+1-x)$.
%
\subsection{Spherical harmonic functions}
%
\subsubsection{Definition}
%
The scalar spherical harmonic functions are defined by
\begin{equation}
  Y_{\ell m}(\theta, \phi) = 
  \sqrt{\frac{4\pi}{2\ell+1} \frac{(\ell-m)!}{(\ell+m)!}} \:
  P_\ell^m(\cos\theta) \:
  \exp{(\im m \phi)}
  ,
\end{equation}
for natural $\ell \in \naturals$, integer $m \in \integers$, $\vert m \vert \leq \ell$, where $P_\ell^m(\cdot)$ are the associated Legendre functions.

Spin raising and lowering operators, $\spinup$ and $\spindown$ respectively, increment and decrement the spin order of a spin $s \in \integers$ function by unity and are defined by
\begin{equation}  
  \spinup \equiv
  -\sin^s \phi 
  \biggl ( 
  \frac{\partial}{\partial \phi} 
  + \frac{\im}{\sin\phi} \frac{\partial}{\partial \theta}
  \biggr)
  \sin^{-s}\phi
\end{equation}
and
\begin{equation}  
  \spindown \equiv 
  -\sin^{-s} \phi 
  \biggl ( 
  \frac{\partial}{\partial \phi} 
  - \frac{\im}{\sin\phi} \frac{\partial}{\partial \theta}
  \biggr)
  \sin^{s}\phi
  ,
\end{equation}
respectively.  
%
The spin-$s$ spherical harmonics can thus be expressed in terms of
the scalar (spin-zero) harmonics through the spin raising and lowering
operators by 
\begin{equation}    
  {}_s Y_{\ell m}(\theta,\phi)
  =
  \biggl[ \frac{(\ell-s)!}{(\ell+s)!} \biggr]^{1/2} 
  \spinup^s
  Y_{\ell m}(\theta,\phi)
  ,
\end{equation}
for $0 \leq s \leq \ell$,
and by 
\begin{equation}    
  {}_s Y_{\ell m}(\theta,\phi)
  =
  (-1)^s
  \biggl[ \frac{(\ell+s)!}{(\ell-s)!} \biggr]^{1/2} 
  \spindown^{-s}
  Y_{\ell m}(\theta,\phi)  
  ,
\end{equation}
for $-\ell \leq s \leq 0$.



\subsubsection{Orthogonality}

The spin spherical harmonics satisfy the orthogonality relation:
\begin{equation}
  \int_\sphere \dx \Omega(\rang) \:
  {}_s Y_{\ell m}(\rang) \:
  {}_s Y_{\ell^\prime m^\prime}^\ast(\rang) 
  = 
  \int_0^{2\pi} \dx \phi \:
  \int_0^{\pi} \dx \theta \sin\theta \:
  {}_s Y_{\ell m}(\theta, \phi) \:
  {}_s Y_{\ell^\prime m^\prime}^\ast(\theta, \phi) 
  = 
  \delta_{\ell \ell^\prime} \:
  \delta_{m m^\prime}
  .
\end{equation}

\subsubsection{Completeness}

The spin spherical harmonics satisfy the completeness relation:
\begin{equation}
  \sum_{\ell=0}^\infty \sum_{m=-\ell}^{\ell}
  {}_s Y_{\ell m}(\rang) \:
  {}_s Y_{\ell m}^\ast(\rang^\prime) 
  = 
  \delta(\rang-\rang^\prime)
  = 
  \frac{1}{\sin\theta} \:
  \delta(\theta-\theta^\prime) \:
  \delta(\phi-\phi^\prime)
\end{equation}


\subsection{Fourier-Bessel basis functions}

\subsubsection{Definition}

Spin Fourier-Bessel basis functions are formed by the produce of the spherical Bessel functions and the spin spherical harmonics:
\begin{equation}
  {}_s Z_{\ell m}(k, \rvec) 
  = \sqrt{\frac{2}{\pi}}\:
  j_\ell(k \rlen) \: {}_s Y_{\ell m}(\rang) 
  ,
\end{equation}
where $k \in \realsnn$.

\subsubsection{Orthogonality}

The (spin) Fourier-Bessel functions satisfy the following orthogonality relation:
\begin{equation}
  \int_\ball \dx^3 \rvec \:
  {}_s Z_{\ell m}(k, \rvec) \:
  {}_s Z_{\ell^\prime m^\prime}^\ast(k^\prime, \rvec)   
   = 
  \frac{1}{k^2} \:
  \delta(k-k^\prime) \:
  \delta_{\ell \ell^\prime} \:
  \delta_{m m^\prime}   
\end{equation}


\subsubsection{Completeness}

The (spin) Fourier-Bessel functions satisfy the following completeness relation:
\begin{align}  
  \sum_{\ell=0}^\infty \sum_{m=-\ell}^{\ell}
  \int_{\realsnn} \dx k k^2 \:
  & {}_s Z_{\ell m}(k, \rvec) \:
  {}_s Z_{\ell m}^\ast(k, \rvec^\prime) \\ 
  & =   
  \frac{2}{\pi} \sum_{\ell=0}^\infty \sum_{m=-\ell}^{\ell} \:
  {}_s Y_{\ell m}(\rang) \:
  {}_s Y_{\ell m}^\ast(\rang^\prime) 
  \int_{\realsnn} \dx k k^2 \: j_\ell(k \rlen) \: j_\ell(k \rlen^\prime) \\
  & = 
  \frac{\pi}{2\rlen^2} \:
  \delta(\rang-\rang^\prime) \:
  \delta(\rlen-\rlen^\prime)\\
  & = 
  \frac{\pi}{2\rlen^2\sin\theta} \:
  \delta(\theta-\theta^\prime) \:
  \delta(\phi-\phi^\prime)
\end{align}


\subsection{Fourier-Laguerre basis functions}
\label{sec-FLag-basis}
%JEC 20 july 16
\subsubsection{definition}
%
Following the definition of the spherical Laguerre functions (Eq.~\ref{eq-sphLag-def}) and the spin spherical harmonics (Eq.~\ref{eq-spin-sph-harm}), one can form a 3D basis, hereafter called "spin Fourier-Laguerre" along the model of the "spin Fourier-Bessel" one, as followed 
\begin{equation}
{}_s{\cal K}_{\ell m n}^\alpha(\rvec, \tau) =  {\cal K}_n^\alpha(r, \tau) {}_sY_{\ell m}(\rang) 
\end{equation}
with $\tau  \in \realsnn$ a scale factor (see Sec.~\ref{sec-Lag-quadrature})
%
\subsubsection{Orthogonality}
%
The orthogonality and completeness of the spin Fourier-Laguerre basis is deduced from the properties on their proper space of the spherical Laguerre functions ${\cal K}_n^\alpha(r, \tau)$ and the spin spherical harmonic functions ${}_sY_{\ell m}(\rang)$. It reads
\begin{align}
\int_\ball \dx^3 \rvec \: {}_s{\cal K}_{\ell m n}^\alpha(\rvec, \tau) {}_s{\cal K}_{\ell^\prime m^\prime n^\prime}^\alpha(\rvec, \tau) &= \left( \int_\realsnn  {\cal K}^\alpha_n(r, \tau) {\cal K}^\alpha_{n^\prime}(r, \tau) r^2 \dx r \right)\nonumber \\  
& \times \left( \int_\sphere \dx \Omega(\rang) \:
  {}_s Y_{\ell m}(\rang) \:
  {}_s Y_{\ell^\prime m^\prime}^\ast(\rang) \right) 
  = \delta_{n n^\prime}\delta_{\ell \ell^\prime}\delta_{m m^\prime}
\end{align}

\section{Fourier-Bessel transforms}



\subsection{Forward transform}

The forward Fourier-Bessel transform reads:

\begin{align}
  {}_s f_{\ell m}(k) 
  & = 
  \sqrt{\frac{2}{\pi}} 
  \int_{\ball} \dx ^3 \rvec \:
  f(\rvec) \:
  j_\ell(k\rlen) \:
  {}_s Y_{\ell m}^\ast(\rang) \\
  & =  
  \int_0^{\pi}\dx \theta 
  \int_0^{2\pi} \dx \phi
  \int_0^\infty \dx \rlen 
  \rlen^2 \:
  f(\rvec) \:
  j_\ell(k\rlen) \:
  {}_s Y_{\ell m}^\ast(\rang) 
\end{align}

\subsection{Inverse transform}

The inverse Fourier-Bessel transform reads:
\begin{align}
  f(\rvec) 
  & =  
  \sqrt{\frac{2}{\pi}} 
  \sum_{\ell m}
  \int_{\realsnn} \dx k k^2 \:
  {}_s f_{\ell m}(k) \:
  j_\ell(k\rlen) \:
  {}_s Y_{\ell m}(\rang) \\
  & =
  \sqrt{\frac{2}{\pi}} 
  \sum_{\ell m}
  \int_0^\infty \dx k k^2 \:
  {}_s f_{\ell m}(k) \:
  j_\ell(k\rlen) \:
  {}_s Y_{\ell m}(\rang) 
\end{align}

\subsection{Quadrature}

\todo{To add\ldots}

\section{Fourier-Laguerre transform}
%JEC 20 July 16
Any spin three dimensional function ${}_sf(\rvec \in L^2(\ball)$ can be decomposed (Forward transform) and  synthesized (Backward transform) on the spin Fourier-Laguerre basis defined in Sec.~\ref{sec-FLag-basis}.
%
\subsection{Forward transform}
%
The forward spin Fourier-Laguerre transform reads
\begin{align}
{}_sf_{\ell m n} &= \int_\ball \dx ^3 \rvec \: {}_sf(\rvec) {}_s{\cal K}_{\ell m n}^{\alpha \ast}(\rvec, \tau)
\nonumber \\
& = \int_0^{\pi}\dx \theta 
  \int_0^{2\pi} \dx \phi
  \int_0^\infty \dx \rlen 
   {}_sY_{\ell m}^\ast(\rang) 
  \rlen^2 \: {}_sf(\rvec) {\cal K}_n^\alpha(r, \tau)
\nonumber \\
&=   \int_0^\infty \dx \rlen  \rlen^2 \: {}_s a_{\ell m}(r)  {\cal K}_n^\alpha(r, \tau)
\label{eq-FLag-Forward-spin}
\end{align}
where one has exhibited the spherical harmonic (intermediate) coefficient defined as
\begin{equation}
{}_s a_{\ell m}(r) =  
\int_0^{\pi}\dx \theta 
  \int_0^{2\pi} \dx \phi  {}_sY_{\ell m}^\ast(\rang) \: {}_sf(\rvec)
\end{equation} 
%
\subsection{Inverse transform}
%
The inverse spin Fourier-Laguerre transform reads
\begin{align}
{}_sf(r,\Omega_r) &= \sum_{\ell m} \sum_{n=0}^{\infty} {}_sf_{\ell mn}\  {}_s{\cal K}_{\ell m n}^\alpha(\rvec, \tau) 
\nonumber \\
 &= \sum_{\ell m}  \sum_{n=0}^{\infty} {}_sf_{\ell mn}\  {\cal K}_n(r,\tau)\ {}_sY_{\ell m}(\rang) 
 \label{eq-FLag-inverse-spin}
\end{align}
%
\subsection{Fourier-Bessel and Fourier-Laguerre relationship}
%
Applying a Backward Fourier-Laguerre transform followed by a Forward Fourier-Bessel transform leads to the relation between Fourier-Bessel and Fourier-Laguerre coefficients
\begin{align}
{}_s f_{\ell m}(k) &= \sqrt{\frac{2}{\pi}}\iint j_\ell(k r) \left( \sum_{\ell^\prime m^\prime}\sum_n {}_s f_{\ell^\prime m^\prime n} {\cal K}^\alpha_n(r,\tau) {}_sY_{\ell^\prime m^\prime}(\rang)\right) {}_sY_{\ell m}^\ast(\rang) \ r^2 \dx r d\Omega(\rang) \\ \nonumber
& = \sum_n {}_s f_{\ell^\prime m^\prime n} J_{\ell n}(k)
\end{align}
with $J_{\ell n}(k)$, the overlapping integral between spherical Bessel function and the spherical Laguerre function, defined by
\begin{equation}
J_{\ell ln}(k) = \sqrt{\frac{2}{\pi}}\int j_l(k r) {\cal K}^\alpha_n(r,\tau)\ r^2 \dx r 
\end{equation}
%
\subsection{Laguerre quadrature}
\label{sec-Lag-quadrature}
%
The radial integral of the Forward transform (Eq.~\ref{eq-FLag-Forward-spin}) can be approximated using the Gauss-Laguerre quadrature at order $N$ as 
\begin{equation}
{}_s f_{\ell m n} = \sum_{i=0}^{N-1} \omega_i\ {}_s a_{\ell m}(r_i)\ {\cal K}^\alpha_n(r_i,\tau)
\label{eq-Lag-analyse-quadra}
\end{equation}
The $\omega_i$ are the quadrature weights and $r_i$ are r-shell radii defined by $r_i = x_i \tau$ with $\tau = R_{max}/x_{N-1}$ with $x_i$ the  nodes of the quadrature, i.e the $i+1$-th root of $N$-th  modified Laguerre polynomial. Such definition of $r_i$ makes $r_i \in [0, R_{max}]$, i.e. the function $f$ is somehow bounded in the radial direction. For completeness the weights are equal to
\begin{equation}
\omega_i = \frac{\Gamma(N+\alpha)}{N(N+1)^2} \frac{r_i e^{r_i}}{[L^{(\alpha)}_{N+1}(r_i)]^2}
\end{equation}
% \appendix
% \section{Some appendix}

% \acknowledgments
% This is the most common positions for acknowledgments. A macro is
% available to maintain the same layout and spelling of the heading.

\end{document}
